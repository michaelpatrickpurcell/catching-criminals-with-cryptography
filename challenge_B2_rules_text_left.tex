\begin{minipage}{6cm}\raggedright
\textsf{\LARGE Overview}\\[1.0ex]

A substitution cipher replaces the char\underline{\textbf{a}}cters in a message but does not rearrange them.\\[1.25ex]

A polya\underline{\textbf{l}}phabetic substitution cipher is a cipher that uses more than one substitution al\underline{\textbf{p}}habet.\\[1.25ex]

The Vigen\'ere cipher is a famous polyalphabetic substitution cipher. The substitution alphabets are determined by a secret keyword agreed upon a\underline{\textbf{h}}ead of time by the mess\underline{\textbf{a}}ge's sender and receiver.\\[3ex] 

\textsf{\LARGE Encryption}\\[1.0ex]

To encrypt a message of length $\ell$ with a keyword:
\begin{enumerate}[leftmargin=*]
	\item Write out the keyword, repeating as necessary to create a key of length $\ell$.\\[1.25ex]
	\item For each character in the key, compute the distance in the alphabet between that character and the letter \texttt{A}.
	\item Shift each character of the message forward by the distance computed for its position in step 2. 
\end{enumerate}
%\textsf{\Large Example}\\[1.0ex]
%
%Suppose you wanted to use the keyword \texttt{GOBLIN} to encrypt the following message:\\[1.25ex]
%
%\texttt{PACK MY RED BOX WITH FIVE DOZEN QUALITY JUGS}.\\[1.25ex]
%
%Test.\\[1.25ex]
%
%Test.\\[1.25ex]
%
%\textbf{Game Design:} Michael Purcell
%
%\textbf{Contact:} ttkttkt@gmail.com
%%\textbf{License:} \doclicenseText
\end{minipage}