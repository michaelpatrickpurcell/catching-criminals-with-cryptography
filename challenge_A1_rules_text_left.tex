\begin{minipage}{6cm}\raggedright
\textsf{\LARGE Overview}\\[1.0ex]

A transp\underline{\textbf{o}}sition cipher reorders the positions of the characters in a message but does not change the characters themse\underline{\textbf{l}}ves.\\[1.25ex]

%You will need a copy of these rules and something to write with.\\[1.125ex]
In a columnar transposition cipher, the message is written out \underline{\textbf{i}}n rows of fixed length, and then read out column-by-column.\\[1.25ex]

Both the width of the rows and the ordering of the columns is determined by a secret keyword agreed upon ahead of time by the message's sender and recei\underline{\textbf{ve}}r.\\[3.0ex]

\textsf{\LARGE Encryption}\\[1.0ex]

To encrypt a message with a keyword of length $n$:
\begin{enumerate}[leftmargin=*]
	\item Write the message in a grid with $n$ columns.\\[1.25ex]
	\item Number the columns of the resulting  grid according to the alphabetical ordering of the letters of the keyword.
	\item Read down the columns of the grid in increasing order of the numbers assigned to the columns in step 2.
\end{enumerate}
%\textsf{\Large Example}\\[1.0ex]
%
%Suppose you wanted to use the keyword \texttt{GOBLIN} to encrypt the following message:\\[1.25ex]
%
%\texttt{PACK MY RED BOX WITH FIVE DOZEN QUALITY JUGS}.\\[1.25ex]
%
%Test.\\[1.25ex]
%
%Test.\\[1.25ex]
%
%\textbf{Game Design:} Michael Purcell
%
%\textbf{Contact:} ttkttkt@gmail.com
%%\textbf{License:} \doclicenseText
\end{minipage}